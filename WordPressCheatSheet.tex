%
%  handleidinghuis
%
%  Created by Niels Doorn on 2010-07-04.
%  Copyright (c) 2010 __MyCompanyName__. All rights reserved.
%
\documentclass[8pt,pagesize,footinclude=false,headinclude=false]{scrartcl}


% Copyright (C) 2008, 2009, 2010 Bert Burgemeister
%
% Permission is granted to copy, distribute and/or modify this
% document under the terms of the GNU Free Documentation License,
% Version 1.2 or any later version published by the Free Software
% Foundation; with no Invariant Sections, no Front-Cover Texts and
% no Back-Cover Texts. For details see file COPYING.
%
\usepackage{amsmath}
\usepackage{amsfonts}
\usepackage{amssymb}
\usepackage{rotating}
\usepackage{graphicx}
\usepackage{multicol}
\usepackage{textcase}
\usepackage{textcomp}
\usepackage{ulem}
\usepackage[usenames,dvips]{color}
\usepackage{suffix}
\usepackage{makeidx}
\usepackage[pagestyles]{titlesec}
\usepackage{titletoc}
%
%%%%%%%%%%%%%%%%%%
% Two font alternatives:
% (A) All (except cover pages) Computer Modern --
%     everything comes from the same sound root; gets about 5% longer
%     than alternative (B) 
\usepackage{type1cm}
\usepackage{exscale}
%%%%%%%%%%%%%%%%%%
% (B) Times mixed with Helvetica --
%     different sources; need scaling as they don't even agree in
%     their concept of height
%\usepackage{mathptmx}
%\usepackage[scaled]{helvet}
%%%%%%%%%%%%%%%%%%
%


% outsourced page dimensions for A4 paper
%\setlength{\paperwidth}{10.5cm}
%\setlength{\paperheight}{29.7cm}
%%\areaset[10mm]{8cm}{29cm}
%\typearea[3mm]{20}

% Use utf-8 encoding for foreign characters
\usepackage[utf8]{inputenc}

\usepackage[dutch]{babel}

\usepackage[a4paper,landscape]{geometry}
%\usepackage[a4paper,hmargin=1cm,vmargin=1cm]{geometry}
\usepackage{multicol}

% Setup for fullpage use
\usepackage[cm]{fullpage}

\usepackage{eurosym}

% Surround parts of graphics with box
\usepackage{boxedminipage}

% Package for including code in the document
\usepackage{listings}

% If you want to generate a toc for each chapter (use with book)
\usepackage{minitoc}

% This is now the recommended way for checking for PDFLaTeX:
\usepackage{ifpdf}

\definecolor{lightgray}{rgb}{0.01,0.01,0.01}

\title{WordPress Cheatsheet}
\author{Niels Doorn}
\date{\today}

\usepackage[bookmarks=true,pdftex,bookmarksopen=true,bookmarksnumbered=true,pdfborder={0 0 0 0}]{hyperref}


\titleformat{\section}{\sffamily\mdseries\slshape}
            {\huge\thesection}{.7em}{\huge}[{\titlerule[0.25pt]}]
            
\titleformat{\subsection}{\sffamily\mdseries\slshape}
            {\Large\thesubsection}{.7em}{\Large}[{\titlerule[0.25pt]}]


\lstset{
        basicstyle=\footnotesize\ttfamily, % Standardschrift
        %numbers=left,               % Ort der Zeilennummern
        numberstyle=\tiny,          % Stil der Zeilennummern
        %stepnumber=2,               % Abstand zwischen den Zeilennummern
        numbersep=5pt,              % Abstand der Nummern zum Text
        tabsize=2,                  % Groesse von Tabs
        extendedchars=true,         %
        breaklines=true,            % Zeilen werden Umgebrochen
        keywordstyle=\bfseries,
		%commentstyle=\em\color{gray},
        frame=trlb,         
%        keywordstyle=[1]\textbf,    % Stil der Keywords
%        keywordstyle=[2]\textbf,    %
%        keywordstyle=[3]\textbf,    %
%        keywordstyle=[4]\textbf,   \sqrt{\sqrt{}} %
        stringstyle=\bfseries\color{gray}, % Farbe der String
        showspaces=false,           % Leerzeichen anzeigen ?
        showtabs=false,             % Tabs anzeigen ?
        xleftmargin=0pt,
        framexleftmargin=0pt,
        framexrightmargin=0pt,
        framexbottommargin=0pt,
        %backgroundcolor=\color{lightgray},
        showstringspaces=false      % Leerzeichen in Strings anzeigen ?        
}
\lstloadlanguages{% Check Dokumentation for further languages ...
         %[Visual]Basic
         %Pascal
         %C
         %C++
         XML,
         HTML,
         Java,
         PHP
}

\hypersetup{
	pdfauthor = {Niels Doorn},
	pdftitle = {WordPress Cheatsheet},
	pdfsubject = {WordPress, PHP, MySQL, Themes, Plugins},
	pdfkeywords = {WordPress, PHP, MySQL, Themes, Plugins},
	pdfcreator = {NielsDoorn/RocVanTwente}
}

\usepackage{lastpage}
\usepackage{fancyhdr}
\pagestyle{fancy}
\rhead{}
\lhead{}
\chead{}
\lfoot{Versie 0.1 - Niels Doorn \copyright~2013}
\cfoot{\url{http://www.nielsdoorn.nl}}
\rfoot{\thepage\ van \pageref{LastPage}}
\renewcommand{\headrulewidth}{0pt}
\renewcommand{\footrulewidth}{0pt}


\usepackage{pst-barcode}

\begin{document}

\ifpdf
\DeclareGraphicsExtensions{.pdf, .jpg, .tif}
\else
\DeclareGraphicsExtensions{.eps, .jpg}
\fi

\begin{multicols*}{4}

\section*{WordPress Cheatsheet}
Dit is een cheatsheet over WordPress Themes en Plugins. Deze versie is nog erg beta.

\section*{Inleiding}
Een WordPress theme is een herbruikbare template voor een WordPress website. Met theme kun je een website vormgeven. De inhoud (tekst, afbeeldingen et cetera) komt dan uit WordPress, de vormgeving wordt bepaald door de theme.

Posts en pages bevatten de inhoud van de site. Bezoekers kunnen reageren op posts en pages door middel van comments. Menu's zorgen voor navigatie binnen en buiten de site. Sidebars bevatten widgets die features aan de site toevoegen, bijvoorbeeld een kalender of een twitter stream.

Plugins voegen features toe aan de site. Dit kunnen widgets zijn maar ook slideshows en galleries of (contact)formulieren. Niet alle plugins voegen zichtbare features toe. Zo heb je ook SEO plugins, anti-spam plugins en plugins om statistieken bij te houden.

\section*{Themes}
Een theme bestaat uit een aantal bestanden die afhankelijk van waar de bezoeker zich op een site bevindt worden gebruikt. Een aantal bestanden moeten altijd aanwezig zijn in een theme. Dat zijn de `style.css' en de `index.php'.
\subsection*{style.css}

\lstinputlisting{code/style.css}

\subsection*{Index.php en page.php}
Het bestand `index.php' is de overzichtspagina van de posts. `page.php' wordt gebruikt voor pages. Beide bevatten aanroepen naar andere onderdelen van de pagina zoals onder andere de `header.php', `sidebar.php' en de `footer.php'. Ook bevatten deze pagina's de zogenaamde loop om de inhoud van de pagina op te vragen.

\noindent Minimale index.php:
\lstinputlisting{code/index.php}

\noindent Voorbeeld van een loop:
\lstinputlisting{code/loop.php}

\subsection*{Header.php, footer.php en sidebar.php}
`Header.php' bevat het begin van je pagina en wordt op ieder onderdeel van je WordPress site gebruikt. Voor `footer.php' geldt hetzelfde maar bevat het einde van je pagina.

\noindent Voorbeeld van header.php:
\lstinputlisting{code/header.php}

\noindent Voorbeeld van een sidebar.php:
\lstinputlisting{code/sidebar.php}

\noindent Minimale footer.php:
\lstinputlisting{code/footer.php}

\subsection*{Functions.php}
In `functions.php' kun je indien nodig code plaatsen die je vanuit je theme wilt aanroepen en inhaken op functionaliteit van WordPress zelf. Hierin registreer je menu's, definieer je shortcodes en schrijf je eigen functies die je binnen je theme wilt gebruiken. `Functions.php' wordt door je theme automatisch aangeroepen.

\noindent Functions.php:
\lstinputlisting{code/functions.php}

\subsection*{screenshot.png}
Screenshot.png bevat een screenshot van je theme. Deze wordt getoond in het admin gedeelte van de site.

\subsection*{Links naar images in HTML}
Met een plaatje in de `images' map van je theme:

\begin{lstlisting}[language=HTML]
<img width="80" height="120" src="<?php bloginfo('template_directory'); ?>/images/logoNoText.svg" />
\end{lstlisting}

\subsection*{Links naar images in css}
Met een plaatje in de `images' map van je theme:

\begin{lstlisting}
.myFancyElement {
	background-image: url('/images/logo.png');
}
\end{lstlisting}

\subsection*{Menu's}
Menu's registreer je in je `functions.php':
\begin{lstlisting}[language=HTML]
register_nav_menu( "hoofdmenu", "Het hoofdmenu" );
\end{lstlisting}

En op de plaats waar je het menu wilt gebruiken:
\begin{lstlisting}[language=HTML]
$opties = array( 
	'container_class' => 'menu', 
	'theme_location' => 'hoofdmenu' 
);
wp_nav_menu( $opties ); 
\end{lstlisting}

\subsection*{Sidebars en widgets}

\begin{lstlisting}[language=HTML]
register_sidebar( array(
	'name' => __( 'Hoofd Sidebar', 'mijntheme' ),
	'id' => 'sidebar-1',
	'before_widget' => '<aside id="%1$s" class="widget %2$s">',
	'after_widget' => "</aside>",
	'before_title' => '<h3 class="widget-title">',
	'after_title' => '</h3>',
) );
\end{lstlisting}

\subsection*{Shortcodes}
Om in een post of page de shortcode [coolfeature] te kunnen gebruiken:

\lstinputlisting{code/shortcode.php}

\clearpage

\section*{Plugins}

\clearpage

\subsection*{Handige websites}
Er zijn veel sites met informatie over WordPress Theme development, hier een paar voorbeelden:
\begin{itemize}
	\item PDF over WordPress Themes in het nederlands \url{http://tiny.cc/6s7ytw}
	\item Webdesigner Wall \url{http://tiny.cc/a17ytw}
	\item WordPress Codex \url{http://tiny.cc/v27ytw}
\end{itemize}

\end{multicols*}
\end{document}